\chapter{Implementation}
Since we are trying to find a program that satisfies an instance of the program synthesis problem, it is natural to use a searching algorithm. We will use a depth-bounded search that uses the idea of iterative deepening. This is because we want our programs to gradually increase in size, so as to not have situations where we apply a function infinitely many times before considering other branches (consider a case where we apply the (\textit{map $\fbox{f}_{\,0}$}) metarule again and again). Also, this assures that the induced program will be the simplest one (\textsc{formalize simple}). This however does not guarantee anything about the complexity of the induced program: to have some claims about the complexity of the induced program, we should use some other cost function.
\\
Intuitively, the nodes in the search tree will be represented by a set of pairs, each describing an induced program. Those pairs will have the form  (\textit{function identifier}, \textit{type declaration}, \textit{body}). We call those pairs \textit{function signatures}. Initially this set will be a singleton containing only the function signature $\{(f_{t}, \theta_{f}, \fbox{m})$, where $f_{t}$ is the target function, $\theta$ is its type declaration and $\fbox{m}$ represents the fact that $f_{t}$ does not yet have a "shape" for its body. The leaves in the search tree will represent a set of complete induced functions that form an induced program. The search will stop once we find a leaf that represents an induced program whose target function satisfies the examples.
\\
We now describe how the searching is done:
\begin{itemize}
\item[1.] If the current induced program has no place-holders, test to see whether the target function satisfies the examples.
\begin{itemize}
\item If it does, we have found a complete program that satisfied the examples.
\item If it does not, we undo the last specialisation step and try step 3 again.
\end{itemize}
\item[2.] If the current induced program has place-holders, select one. Suppose we consider the function signature $\{(f, \theta_{f}, E_{f})$, where $E_{f}$ has at least a place-holder. The specialisation step should be described based on the form of the place-holder:
\begin{itemize}
\item $\fbox{m}$: then the specialisation step will look like $E_{f} \xrightarrow[M]{\fbox{m}} E'_{f}$, where $M$ is a metarule as described in definition 3.
\item $\fbox{f}_{\,i}$: then the specialisation step will look like $E_{f} \xrightarrow[\fbox{f}_{\,i}]{\fbox{f}} E'_{f}$, where $f$ is either a first-order background function, an existing induced function or a to be added induced function, tried in this order; we expand on what we mean by "to be added": after trying a first-order background function and an existing induced function, we could only go forward by saying that our place-holder will be substituted by another function which we will synthesize in the future; to do this we add a new element to the set of induced functions $(g, \theta_{g}, \fbox{m})$.
\end{itemize}
\end{itemize}
